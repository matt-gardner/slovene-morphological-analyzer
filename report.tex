\documentclass[11pt,letterpaper]{article}
\usepackage{fullpage}
\usepackage[pdftex]{graphicx}
\usepackage{amsfonts,eucal,amsbsy,amsopn,amsmath}
\usepackage{url}
\usepackage[sort&compress]{natbib}
\usepackage{natbibspacing}
\usepackage{latexsym}
\usepackage{wasysym} 
\usepackage{rotating}
\usepackage{fancyhdr}
\DeclareMathOperator*{\argmax}{argmax}
\DeclareMathOperator*{\argmin}{argmin}
\usepackage[dvipsnames,usenames]{color}
\usepackage{multicol}
\definecolor{orange}{rgb}{1,0.5,0}
\usepackage{caption}
\renewcommand{\captionfont}{\small}
\setlength{\oddsidemargin}{-0.04cm}
\setlength{\textwidth}{16.59cm}
\setlength{\topmargin}{-0.04cm}
\setlength{\headheight}{0in}
\setlength{\headsep}{0in}
\setlength{\textheight}{22.94cm}
\newcommand{\ignore}[1]{}
\newenvironment{enumeratesquish}{\begin{list}{\addtocounter{enumi}{1}\arabic{enumi}.}{\setlength{\itemsep}{-0.25em}\setlength{\leftmargin}{1em}\addtolength{\leftmargin}{\labelsep}}}{\end{list}}
\newenvironment{itemizesquish}{\begin{list}{\setcounter{enumi}{0}\labelitemi}{\setlength{\itemsep}{-0.25em}\setlength{\labelwidth}{0.5em}\setlength{\leftmargin}{\labelwidth}\addtolength{\leftmargin}{\labelsep}}}{\end{list}}

\bibpunct{(}{)}{;}{a}{,}{,}
\newcommand{\nascomment}[1]{\textcolor{blue}{\textbf{[#1 --NAS]}}}


\pagestyle{fancy}
\lhead{}
\chead{}
\rhead{}
\lfoot{}
\cfoot{\thepage~of \pageref{lastpage}}
\rfoot{}
\renewcommand{\headrulewidth}{0pt}
\renewcommand{\footrulewidth}{0pt}


\title{11-712:  NLP Lab Report\\Slovene}
\author{Matt Gardner}
\date{April 26, 2013}

\begin{document}
\maketitle
\begin{abstract}
\nascomment{one paragraph here summarizing what the paper is about}
\end{abstract}

\nascomment{brief introduction}

\section{Basic Information about Slovene}

Slovene is a Slavic language spoken in the country of Slovenia.  It is spoken
natively by around 2 million people, most of whom live in Slovenia.  Slovene is
one of the official languages of the European Union, so there is a moderate
amount of data available for computational linguists.  Slovene's closest
relative is Serbo-Croatian, being mutually intelligible with some variants of
Croatian.  As with many Slavic languages, Slovene is highly inflected and has
relatively free word order~\citep{wikipedia-slovene}.

\section{Past Work on the Morphology of Slovene}

There has been some work on the Slovene language by linguists.  There is a
grammar written by \cite{greenberg-2006-slovene-grammar} that gives some 75
pages to the morphology of the language.  There is also a more comprehensive
grammar written by \cite{herrity-2000-slovene-grammar}.  There are a few other
papers on Slovene morphology to be found in English by searching Google scholar
(e.g., \cite{bidwell-1969-outline-slovene-morphology}), and more that is
written in Slovene.

As far as work in computational linguists on Slovene, a small body of
literature exists.  Toma\v{z} Erjavec, from the Jo\v{z}ef Stefan Institute,
published a description of an early morphological analyzer for Slovene.  The
paper is only a two-and-a-half page, very high-level description, and the
system was implemented in Prolog on a VAX
machine~\citep{erjavec-1990-slovene-analyzer}.  Erjavec also was a main
contributor to the Slovene Dependency Treebank, which includes morphological
information and was a part of the CoNLL-X Shared Task on multilingual
dependency parsing in 2006~\citep{dvzeroski-2006-sdt}.

More recently, there was a Slovene morphological analyzer released on
SourceFourge in May of 2012.  This analyzer uses a maximum entropy model and
achieves 92.5\% accuracy~\citep{grcar-2012-obeliks}.  The system (called
Obeliks) was trained on a corpus of five hundred thousand words.

\section{Available Resources}

The available resources for work on the Slovene language consist largely of
things mentioned in the previous section.  I already listed reference grammars,
the Slovene dependency treebank, and a corpus available with the Obeliks tool.
In addition, there is a Slovene lexicon available with the MULTEXT-East
dataset~\citep{erjavec-1998-slovene-lexicon}; I have requested access to the
lexicon and hope to have it soon.  Assuming I get access to the lexicon, I will
use the lexicon as I develop the analyzer, and I will use the morphology labels
from the Slovene Dependency Treebank and from the dataset released with the
Obeliks system as my two test sets.  These two datasets have hundreds of
thousands of tokens (though I have not yet counted the number of word types),
and the lexicon has over 15,000 lemmas, with full inflectional paradigms for
each lemma.

\section{Survey of Phenomena in Slovene}

For the phenomena my morphological analyzer will cover, I will follow exactly
the morphosyntactic specifications given by~\cite{erjavec-mds}.  This
specification gives twelve parts of speech.  Each part of speech has a set of
features or attributes that are marked in words of that part of speech.  These
are the parts of speech, with their features:

\begin{itemize}
  \item \textbf{Noun}: type (common / proper), gender
    (masculine / feminine / neuter), number (singular / dual / plural),
    case (nominative / genitive / dative / accusative / locative /
    instrumental), animate (yes / no)
  \item \textbf{Verb}: type (main / auxiliary), aspect (perfective /
    progressive / biaspectual), form (infinitive / supine / participle /
    present / future / conditional / imperative), person (first / second /
    third), number, gender, negative (yes / no)
  \item \textbf{Preposition}: case (specifies the case it governs, not the case
    it is inflected for)
  \item \textbf{Conjunction}: type (coordinating / subordinating)
  \item \textbf{Adjective}: type (general / possessive / participle), degree
    (positive / comparative / superlative), gender, number, case, definiteness
    (yes / no)
  \item \textbf{Pronoun}: type (personal / possessive / demonstrative /
    relative / reflexive / general / interrogative / indefinite / negative),
    person, gender, number, case, owner number (for possessive pronouns), owner
    gender, clitic (yes / bound)
  \item \textbf{Adverb}: type (general / participle), degree
  \item \textbf{Numeral}: form (digit / roman / letter), type (cardinal /
    ordinal / pronominal / special), gedner, number, case, definiteness
  \item \textbf{Particle}: (no features)
  \item \textbf{Interjection}: (no features)
  \item \textbf{Abbreviation}: (no features)
  \item \textbf{Residual}: type (foreign / typo / program)
\end{itemize}

This list is roughly in the order I would assign them priority.  I doubt that I
would be able to handle open class abbreviations or interjections, or the
loosely-defined ``residual'' part of speech.  The closed classes should be
relatively easy, as I have a large corpus that has been labeled with these
morphosyntactic descriptions, so I will only need to produce a few small
lexicons for these words.  I will probably not try to handle common nouns vs.
proper nouns at first, nor to distinguish main verb vs. auxiliary verb (which
depends on context).  Aspect of verbs will also be difficult, and so I will
save it for later, and there are a few features listed above that only rarely
show up in word forms (like noun animacy and verb negativity) which could be
ignored at the beginning.

\section{Initial Design}

\section{System Analysis on Corpus A}

\section{Lessons Learned and Revised Design}

\section{System Analysis on Corpus B}

\section{Final Revisions}

\section{Future Work}





\bibliographystyle{plainnat}
\bibliography{bib}
\label{lastpage}
\end{document}
