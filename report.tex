\documentclass[11pt,letterpaper]{article}
\usepackage{fullpage}
\usepackage[pdftex]{graphicx}
\usepackage{amsfonts,eucal,amsbsy,amsopn,amsmath}
\usepackage{url}
\usepackage[sort&compress]{natbib}
\usepackage{natbibspacing}
\usepackage{latexsym}
\usepackage{wasysym} 
\usepackage{rotating}
\usepackage{fancyhdr}
\DeclareMathOperator*{\argmax}{argmax}
\DeclareMathOperator*{\argmin}{argmin}
\usepackage[dvipsnames,usenames]{color}
\usepackage{multicol}
\definecolor{orange}{rgb}{1,0.5,0}
\usepackage{caption}
\renewcommand{\captionfont}{\small}
\setlength{\oddsidemargin}{-0.04cm}
\setlength{\textwidth}{16.59cm}
\setlength{\topmargin}{-0.04cm}
\setlength{\headheight}{0in}
\setlength{\headsep}{0in}
\setlength{\textheight}{22.94cm}
\newcommand{\ignore}[1]{}
\newenvironment{enumeratesquish}{\begin{list}{\addtocounter{enumi}{1}\arabic{enumi}.}{\setlength{\itemsep}{-0.25em}\setlength{\leftmargin}{1em}\addtolength{\leftmargin}{\labelsep}}}{\end{list}}
\newenvironment{itemizesquish}{\begin{list}{\setcounter{enumi}{0}\labelitemi}{\setlength{\itemsep}{-0.25em}\setlength{\labelwidth}{0.5em}\setlength{\leftmargin}{\labelwidth}\addtolength{\leftmargin}{\labelsep}}}{\end{list}}

\bibpunct{(}{)}{;}{a}{,}{,}
\newcommand{\nascomment}[1]{\textcolor{blue}{\textbf{[#1 --NAS]}}}


\pagestyle{fancy}
\lhead{}
\chead{}
\rhead{}
\lfoot{}
\cfoot{\thepage~of \pageref{lastpage}}
\rfoot{}
\renewcommand{\headrulewidth}{0pt}
\renewcommand{\footrulewidth}{0pt}


\title{11-712:  NLP Lab Report\\Slovene}
\author{Matt Gardner}
\date{April 26, 2013}

\begin{document}
\maketitle
\begin{abstract}
\nascomment{one paragraph here summarizing what the paper is about}
\end{abstract}

\nascomment{brief introduction}

\section{Basic Information about Slovene}

Slovene is a Slavic language spoken in the country of Slovenia.  It is spoken
natively by around 2 million people, most of whom live in Slovenia.  Slovene is
one of the official languages of the European Union, so there is a moderate
amount of data available for computational linguists.  Slovene's closest
relative is Serbo-Croatian, being mutually intelligible with some variants of
Croatian.  As with many Slavic languages, Slovene is highly inflected and has
relatively free word order~\citep{wikipedia-slovene}.

\section{Past Work on the Morphology of Slovene}

There has been some work on the Slovene language by linguists.  There is a
grammar written by \cite{greenberg-2006-slovene-grammar} that gives some 75
pages to the morphology of the language.  There is also a more comprehensive
grammar written by \cite{herrity-2000-slovene-grammar}.  There are a few other
papers on Slovene morphology to be found in English by searching Google scholar
(e.g., \cite{bidwell-1969-outline-slovene-morphology}), and more that is
written in Slovene.

As far as work in computational linguists on Slovene, a small body of
literature exists.  One main contributor to this research is Toma\v{z} Erjavec
at the Jo\v{z}ef Stefan Institute, who in 1990 published a description of an
early morphological analyzer for Slovene.  The paper is only a two-and-a-half
page, very high-level description, and the system was implemented in Prolog on
a VAX machine~\citep{erjavec-1990-slovene-analyzer}.  Erjavec also was a main
contributor to the Slovene Dependency Treebank, which includes morphological
information and was a part of the CoNLL-X Shared Task on multilingual
dependency parsing in 2006~\citep{dvzeroski-2006-sdt}.

More recently, there was a Slovene morphological analyzer released on
SourceFourge in May of 2012.  This analyzer uses a maximum entropy model and
achieves 92.5\% accuracy~\citep{grcar-2012-obeliks}.  The system (called
Obeliks) was trained on a corpus of five hundred thousand words.

\section{Available Resources}

\nascomment{include discussion of your corpora}

\section{Survey of Phenomena in Slovene}

\section{Initial Design}

\section{System Analysis on Corpus A}

\section{Lessons Learned and Revised Design}

\section{System Analysis on Corpus B}

\section{Final Revisions}

\section{Future Work}





\bibliographystyle{plainnat}
\bibliography{bib}
\label{lastpage}
\end{document}
